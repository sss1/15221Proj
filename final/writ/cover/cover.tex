\documentclass[11pt,english]{article}
\usepackage{babel}
\usepackage[margin=1.2in]{geometry}
\usepackage{amsmath, amsfonts, amsthm, amssymb}

%%%%%%%%%%%%%%%%%%%%%%%%HEADER%%%%%%%%%%%%%%%%%%%%%%%%%%%%%
\title{
{\normalsize \bf Technical Communication for Computer Scientists\\
Summer 2013}\\
\vspace{2cm}
{\bf Final Report:\\The Movie-Chain-Runner Project}}
\author{
}
%%%%%%%%%%%%%%%%%%%%%%%%%%%%%%%%%%%%%%%%%%%%%%%%%%%%%%%%%%%

\begin{document}
% TITLE PAGE
\pagenumbering{gobble} % skip page-numbering on the Title Page
\begin{titlepage}
\maketitle
\vfill
\begin{center}
Submitted to\\
Thomas M. Keating\\
Assistant Teaching Professor\\
Computer Science Department\\
Carnegie Mellon University\\
\vspace{1cm}
Prepared and Submitted by\\
\vspace{0.5cm}

\begin{tabular}{cc}
Shashank Singh \hspace{2cm} & Eugene Scanlon \\
\texttt{sss1@andrew.cmu.edu} \hspace{2cm} & \texttt{escanlon@cmu.edu}
\end{tabular}

\vspace{2cm}
{\bf Abstract}\\
\end{center}
We attempted to solve the Movie-Chain-Runner Problem, a computational problem
equivalent to the well-known Longest Path Problem. We designed and used Python
to implement several algorithms for the problem, and include our best solution.
In the end, we accomplished our original goal of constructing a movie chain of
at least 300 titles. We also performed some analyses of both problem and our
algorithms, and use them here to discuss approaches to the problem. Finally, we
overview and reflect on our group's approach to and success.

\end{titlepage}
\end{document}
