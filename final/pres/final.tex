%%%%%%%%%%%%%% COPYRIGHT INFORMATION %%%%%%%%%%%%%%
% Template by
% Author: Sascha Frank 
% Nov. 2006
% University Freiburg 
% www.informatik.uni-freiburg.de/~frank/
% Distributed freely for non-commercial use

%%%%%%%%%%%%%%%%%%%%% PREAMBLE %%%%%%%%%%%%%%%%%%%%%
\documentclass{beamer}
\usepackage{beamerthemeshadow}
\usepackage{lmodern}
\usepackage{pifont}
\usepackage{color}

% include frame number on each frame
\setbeamertemplate{footline}{\quad \insertframenumber/\inserttotalframenumber}
\newcommand{\cmark}{\ding{51}} % check-mark
\newcommand{\xmark}{\ding{55}} % X-mark

%%%%%%%%%%%%%%%%%%%%% DOCUMENT %%%%%%%%%%%%%%%%%%%%%
\begin{document}
\title{Movie-Chain-Runner Problem:\\ Final Report}
\author[Singh,Zong]{Shashank Singh \and Jimmy Zong}
\date{June 27, 2013}
\frame{\titlepage}

\frame{\frametitle{Outline}\tableofcontents}

\section{Team Members}
\frame{\frametitle{Team Members}
\begin{itemize}
\item Sung Uk Ryu
\item Eugene Scanlon
\item Shashank Singh
\item Jimmy Zong
\end{itemize}
}

\section{Problem Introduction}
\frame{\frametitle{The Problem}
\begin{block}{The Problem}
Find the ``longest'' list of overlapping titles in a list of movie titles.
\end{block}

For Example: In the list
\begin{itemize}
\item Day of the Dead
\item Live and Let Die
\item Dead Poets' Society
\item Die Another Day
\item The Last Samurai
\end{itemize}
the ``longest'' chain is
\begin{center}
``Live and Let Die Another Day of the Dead Poets' Society.''
\end{center}
}

\frame{\frametitle{The Problem}
\begin{itemize}
\item Equivalent to finding a Longest Path in a directed graph
\pause
\item The Longest Path Problem is NP-hard, meaning that there is no efficient
algorithm for solving it on a large graph
\end{itemize}
}

\frame{\frametitle{Previous Attempts}
\begin{itemize}
\item Summer 2010   --  255 titles
\item Fall 2010     --  271 titles (845 words)
\item Summer 2011   --  311 titles (997 words)
\item Fall 2011     --  323 titles (1030 words)
\item Spring 2012   --  327 titles (1055 words)
\end{itemize}
}

\section{Approach}
\frame{\frametitle{Approach}
\begin{itemize}
\item Algorithms
\item Project Timeline (Gantt Chart)
\end{itemize}
}

\subsection{Algorithms}
\frame{\frametitle{Algorithms}
\begin{enumerate}
\item Brute Force
\begin{itemize}
\item Tried running on 16 GHC machines for 15 hours
\item Constructed chain of 247 titles
\item Progress slowed exponentially
\end{itemize}
\pause
\item Acyclic Subgraphs
\begin{itemize}
\item A poly-time topo-sort algorithm is known for acyclic graphs
\item Try to find acyclic subgraphs
\item Too many cycles -- took too long to generate subgraphs
\end{itemize}
\pause
\item Working backward
\begin{itemize}
\item Stuck at 247 titles using brute force
\item Reverse graph edges and add to the beginning of the chain
\item Work in progress -- currently managed 274 titles
\end{itemize}
\end{enumerate}
}

\section{Evaluation}
\frame{\frametitle{Evaluation}
\begin{enumerate}
\item Length of the longest chain we find
\pause
\item Compare performance of a different algorithms
\begin{itemize}
\item ideally, decide on a ``best'' algorithm for the problem
\end{itemize}
\pause
\item Predict runtime for entire computation by solving tractable subproblems
and extrapolating
\end{enumerate}
}

\section{Benefits}
\frame{\frametitle{Benefits}
\onslide<1->{The members of our group will gain experience with}
\onslide<2->{
\begin{table}
\centering
\begin{tabular}{c|c}
Python                                              & \onslide<3->{\cmark}  \\
\hline
working as a group toward a common goal             & \onslide<4->{\cmark}  \\
\hline
handling and processing a large data set            & \onslide<5->{\cmark}  \\
\hline
graph algorithms                                    & \onslide<6->{\cmark
                                                        $\times \frac12$}   \\
\hline
approximation algorithms for an NP-hard problem     & \onslide<7->{\xmark}
\end{tabular}
\caption{Benefits of our project}
\label{tab:benefits}
\end{table}
}

}

\section{Summary}
\frame{\frametitle{Summary}\tableofcontents}

\frame{\frametitle{Sources}
\begin{itemize}
\item Gantt Charts created using software from the Gantt Project
\begin{itemize}
\item http://www.ganttproject.biz/ (accessed June 4, 2013)
\end{itemize}
\item Git repository hosted on GitHub
\begin{itemize}
\item https://github.com/
\end{itemize}
\end{itemize}
}
\end{document}
